%%%%%%%%%%%%%%%%%%%%%%%%%%%%%%%%%%%%%%%%%
% Medium Length Graduate Curriculum Vitae
% LaTeX Template
% Version 1.1 (9/12/12)
%
% This template has been downloaded from:
% http://www.LaTeXTemplates.com
%
% Original author:
% Rensselaer Polytechnic Institute (http://www.rpi.edu/dept/arc/training/latex/resumes/)
%
% Important note:
% This template requires the res.cls file to be in the same directory as the
% .tex file. The res.cls file provides the resume style used for structuring the
% document.
%
%%%%%%%%%%%%%%%%%%%%%%%%%%%%%%%%%%%%%%%%%

%----------------------------------------------------------------------------------------
%	PACKAGES AND OTHER DOCUMENT CONFIGURATIONS
%----------------------------------------------------------------------------------------

\documentclass{res} % Use the res.cls style, the font size can be changed to 11pt or 12pt here

\usepackage{helvet} % Default font is the helvetica postscript font

\begin{document}

%----------------------------------------------------------------------------------------
%	NAME AND ADDRESS SECTION
%----------------------------------------------------------------------------------------

\moveleft.5\hoffset\centerline{\large\bf Dave Sayles}

\moveleft.5\hoffset\centerline{sayles.dave@gmail.com\hspace{64pt}www.github.com/dgs3}

 
\moveleft1.5\hoffset\vbox{\hrule width\resumewidth height 1pt}\smallskip % Horizontal line after name; adjust line thickness by changing the '1pt'


%----------------------------------------------------------------------------------------

\begin{resume}

%----------------------------------------------------------------------------------------
%	EDUCATION SECTION
%----------------------------------------------------------------------------------------

\section{Education}

{\sl Bachelor of Science,} Computer Science and Digital Art \\
Union College, Schenectady, NY, 12308
 
%----------------------------------------------------------------------------------------
%	COMPUTER SKILLS SECTION
%----------------------------------------------------------------------------------------

%----------------------------------------------------------------------------------------
%	PUBLICATIONS SECTION
%----------------------------------------------------------------------------------------
 
\section{Publications}
J. Rieffel and D. Sayles. EvoFab: A Fully Embodied Evolutionary Fabricator.  \\In: 9th International Converence on Evolvable Systems, Sep 6 2010 \\

%----------------------------------------------------------------------------------------
%	PROFESSIONAL EXPERIENCE SECTION
%----------------------------------------------------------------------------------------
 
\section{Experience}

{\sl Director - Internal Tools} \hfill 2025 -- Present \\
MongoDB, New York, NY
\begin{itemize}
\item[-] Built a leadership pipeline by promoting senior engineers into engineering managers, tech leads, and staff engineers, and coaching them to independently own teams and roadmaps.
\item[-] Defined and rolled out a Tech Lead role that created a non-manager growth path for senior ICs and let teams scale without adding new engineering managers; the model was adopted across the wider department.
\item[-] Took a legacy Go-To-Market web app from a 12-engineer, fire-fighting team into true maintenance mode by aggressively burning down tech debt and simplifying ownership.
\item[-] Partnered with Product, Tech Leads, and Support to drive support volume from multiple engineering escalations per day down to roughly 1--3 per week by improving tooling, documentation, and workflows.
\item[-] Directly led a small cross-functional pod that delivered an agentic Atlas cluster sizing tool for sales, turning vague sales feedback into concrete, automatable recommendations.
\item[-] Balanced a 14-engineer org against department-wide priorities—routinely triaged and re-scoped work and staffing so teams stayed focused and rightsized on the highest-impact initiatives.
\item[-] Pushed AI usage by championing cursor usage, holding hackathons, and building tooling to use Claude Code to automate opening PRs from JIRA ticket descriptions.
\item[-] Worked with Product and Tech Leads to build out team roadmaps, goals, sprints, and JIRA hygiene.
\end{itemize}

{\sl Senior Lead - Internal Tools} \hfill 2022 -- 2025 \\
MongoDB, New York, NY
\begin{itemize}
\item[-] Rescued a top-10 company-initiative web app from chronic failures and zero observability to a responsive, instrumented, and trusted system that shipped reliably enough to justify continued GTM investment.
\item[-] Replaced a painful weekly, manual QA release process with automated integration tests in the release pipeline, enabling safe on-demand deployments.
\item[-] Doubled team size and split engineers into focused pods, allowing parallel feature work without fragmenting ownership or lowering quality.
\item[-] Direectly managed a team of 5 engineers, plus 1 team lead and a sub-team of 4, working with Product to define roadmap, instrumentation strategy, and execution model.
\item[-] Instituted a cadence of 6-week growth conversations and other feedback loops. Lead to painless yearly reviews, where engineers were never surprised by wins or growth areas.
\item[-] Reworked agile ceremonies to maximize signal and minimize time: standups shrank from 30 minutes to 5, and sprint reset dropped from roughly an hour to about 15 minutes.
\end{itemize}

{\sl Staff Engineer - Internal Tools} \hfill 2022 -- 2022 \\
MongoDB, New York, NY
\begin{itemize}
\item[-] Brought agile and observability (Prometheus, alerting, dashboards, error collation) to teams that had neither; deployments moved from chronic failure to stable, debuggable, and alert-driven.
\item[-] Led as technical owner and product manager: built project roadmaps, spun up and drove alignment for team processes, set priorities, and ran sprint planning with explicit quarterly goals to improve tooling and developer experience.
\end{itemize}

{\sl Senior Engineer - Internal Tools} \hfill 2020 -- 2022 \\
MongoDB, New York, NY
\begin{itemize}
\item[-] Raised the bar on Operational Excellence by implementing and socializing Modern CI/CD and Git workflows, Idiomatic Python, and Integration Testing.
\item[-] Championed admin tooling that removed most direct human interaction with production databases, allowing more engineers to resolve customer issues without dangerous write access.
\item[-] Architected and implemented several web applications for GoToMarket segments and Access Management that are still in use today by the company. 
\end{itemize}

{\sl Senior Engineer} \hfill 2019 -- 2020 \\
Neverware, Inc, New York, NY
\begin{itemize}
\item[-] Neverware was acquired by Google in 2020.
\item[-] Worked on CloudReady OS, a custom fork of Chrome OS; modified system services to improve compatibility across hardware types.
\end{itemize}

{\sl Director of Engineering} \hfill July 2014 - Present \\
Neverware, Inc, New York, NY
\begin{itemize}
\item[-] Reported directly to the CEO during a major company pivot; staffed and organized the engineering team for CloudReady (a Chrome OS fork), ran the build-out, and shepherded the product through launch and multiple iteration cycles. CloudReady still exists today as FlexOS.
\item[-] Grew engineering team from 3 engineers to 7, added 2 qa engineers, and technical project manager.
\item[-] Spent about ~30\% of time working on engineering tasks.
\end{itemize}

{\sl Software Engineer} \hfill Feb 2014 - July 2014 \\
Neverware, Inc, New York, NY
\begin{itemize}
\item[-] Raised the bar on Operational Excellence by implementing and socializing Modern CI/CD and Git workflows, Idiomatic Python, code review, and automated testing.
\item[-] Stabilized opaquely failing product through debugging, admin tooling, and telemetry.
\item[-] Engineer on PCReady, a localized virtualization solution for education customers that delivered ephemeral VMs.
\end{itemize}

{\sl Software Engineer} \hfill Apr 2012 - Feb 2014 \\
Makerbot Industries, Brooklyn, NY
\begin{itemize}
\item[-] MakerBot was acquired by Stratasys in 2013.
\item[-] Primary developer on the \textit{MakerWare} stack: architected \textit{kaiten} (embedded Python server for Desktop 3D printers), backend \textit{conveyor} (job dispatch to slicer and driver), and \textit{s3g} printer driver.
\end{itemize}

%----------------------------------------------------------------------------------------

\end{resume}
\end{document}
