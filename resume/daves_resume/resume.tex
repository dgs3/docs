%%%%%%%%%%%%%%%%%%%%%%%%%%%%%%%%%%%%%%%%%
% Medium Length Graduate Curriculum Vitae
% LaTeX Template
% Version 1.1 (9/12/12)
%
% This template has been downloaded from:
% http://www.LaTeXTemplates.com
%
% Original author:
% Rensselaer Polytechnic Institute (http://www.rpi.edu/dept/arc/training/latex/resumes/)
%
% Important note:
% This template requires the res.cls file to be in the same directory as the
% .tex file. The res.cls file provides the resume style used for structuring the
% document.
%
%%%%%%%%%%%%%%%%%%%%%%%%%%%%%%%%%%%%%%%%%

%----------------------------------------------------------------------------------------
%	PACKAGES AND OTHER DOCUMENT CONFIGURATIONS
%----------------------------------------------------------------------------------------

\documentclass{res} % Use the res.cls style, the font size can be changed to 11pt or 12pt here

\usepackage{helvet} % Default font is the helvetica postscript font

\begin{document}

%----------------------------------------------------------------------------------------
%	NAME AND ADDRESS SECTION
%----------------------------------------------------------------------------------------

\moveleft.5\hoffset\centerline{\large\bf Dave Sayles}

\moveleft.5\hoffset\centerline{sayles.dave@gmail.com\hspace{64pt}www.github.com/dgs3}

 
\moveleft1.5\hoffset\vbox{\hrule width\resumewidth height 1pt}\smallskip % Horizontal line after name; adjust line thickness by changing the '1pt'


%----------------------------------------------------------------------------------------

\begin{resume}

%----------------------------------------------------------------------------------------
%	EDUCATION SECTION
%----------------------------------------------------------------------------------------

\section{Education}

{\sl Bachelor of Science,} Computer Science and Digital Art \\
Union College, Schenectady, NY, 12308
 
%----------------------------------------------------------------------------------------
%	COMPUTER SKILLS SECTION
%----------------------------------------------------------------------------------------

%----------------------------------------------------------------------------------------
%	PUBLICATIONS SECTION
%----------------------------------------------------------------------------------------
 
\section{Publications}
J. Rieffel and D. Sayles. EvoFab: A Fully Embodied Evolutionary Fabricator.  \\In: 9th International Converence on Evolvable Systems, Sep 6 2010 \\

%----------------------------------------------------------------------------------------
%	PROFESSIONAL EXPERIENCE SECTION
%----------------------------------------------------------------------------------------
 
\section{Experience}

{\sl Director - Internal Tools} \hfill 2025 -- Present \\
MongoDB \\
\begin{itemize}
\item[-] Built a leadership pipeline by promoting senior engineers into engineering managers, tech leads, and a staff engineer, and coaching them to independently own teams and roadmaps.
\item[-] Defined and rolled out a Tech Lead role that created a non-manager growth path for senior ICs and let teams scale without adding new engineering managers; the model was adopted across the wider department.
\item[-] Took a legacy Go-To-Market web app from a 12-engineer, fire-fighting team into true maintenance mode by aggressively burning down tech debt and simplifying ownership.
\item[-] Partnered with Product, Tech Leads, and Support to drive support volume from multiple engineering escalations per day down to roughly 1--3 per week by improving tooling, documentation, and workflows.
\item[-] Directly led a small cross-functional pod that delivered an agentic Atlas cluster sizing tool for sales, turning vague sales feedback into concrete, automatable recommendations.
\item[-] Built an internal AI service that converts JIRA ticket descriptions into draft GitHub pull requests, removing boilerplate and shortening the path from idea to reviewed code.
\item[-] Balanced a 14-engineer org against department-wide priorities—routinely triaged and re-scoped work and staffing so teams stayed focused and rightsized on the highest-impact initiatives.
\end{itemize}

{\sl Senior Lead - Internal Tools} \hfill 2022 -- 2025 \\
MongoDB \\
\begin{itemize}
\item[-] Rescued a top-10 company-initiative web app from chronic failures and zero observability to a responsive, instrumented, and trusted system that shipped reliably enough to justify continued GTM investment.
\item[-] Replaced a painful weekly, manual QA release process with automated integration tests in the release pipeline, enabling safe on-demand deployments.
\item[-] Doubled team size and split engineers into focused pods, allowing parallel feature work without fragmenting ownership or lowering quality.
\item[-] Direectly managed a team of 5 engineers, plus 1 team lead and a sub-team of 4, working with Product to define roadmap, instrumentation strategy, and execution model.
\item[-] Instituted a cadence of 6-week growth conversations and other feedback loops. Lead to painless yearly reviews, where engineers were never surprised by wins or growth areas.
\item[-] Reworked agile ceremonies to maximize signal and minimize time: standups shrank from 30 minutes to 5, and sprint reset dropped from roughly an hour to about 15 minutes.
\end{itemize}

{\sl Staff Engineer - Internal Tools} \hfill 2020 -- 2022 \\
MongoDB \\
\begin{itemize}
\item[-] Introduced best-in-class agile practices and deep application telemetry to teams that previously had neither, raising both delivery speed and operational confidence.
\item[-] Rolled out Prometheus metrics, alerting, dashboards, and error collation, moving deployments from near-constant ambiguity and failure to a stable, debuggable state with clear alerts.
\item[-] Led as technical owner: built project roadmaps, set priorities, and ran sprint planning with explicit quarterly goals to improve tooling and developer experience.
\end{itemize}

{\sl Senior Engineer - Internal Tools} \hfill 2019 -- 2020 \\
MongoDB \\
\begin{itemize}
\item[-] Raised the bar on Operational Excellence by implementing and socializing Modern CI/CD and Git workflows, Idiomatic Python, and Integration Testing.
\item[-] Championed single-action deployments by partnering with infrastructure teams to design and ship pipelines triggered by \texttt{git} actions, replacing a brittle, multi-step deploy process.
\item[-] Built admin tooling that removed most direct human interaction with production databases, allowing more engineers to resolve customer issues without dangerous write access.
\item[-] Architected and implemented several web applications still in use today by the company. 
\end{itemize}

{\sl Director of Engineering} \hfill July 2014 - Present \\
Neverware, Inc, New York, NY
\begin{itemize}
\item[-] Work with the CEO and Director of Product to determine product future
\item[-] Grew engineering team from 3 engineers to 7, added 2 qa engineers, and technical project manager.
\item[-] Work with engineers and other department heads to determine feature
priority and scope
\item[-] Frequently act as level 2 support to diagnose potential bugs
\item[-] Spent about ~30\% of time working on engineering tasks
\end{itemize}

{\sl Software Engineer} \hfill Feb 2014 - July 2014 \\
Neverware, Inc, New York, NY
\begin{itemize}
\item[-] Engineer on PCReady, a localized virtualization solution for education customers that delivered ephemeral VMs.
\item[-] Refactored huge amounts of code to allow clients to connect to VMs using the "Spice" protocol.
\item[-] Standardized client/server interface, git usage, and coding style.
\end{itemize}

{\sl Software Engineer} \hfill Apr 2012 - Feb 2014 \\
Makerbot Industries, Brooklyn, NY
\begin{itemize}
\item[-] Architected and wrote \textit{kaiten}, the embedded python server that schedules and organizes processes such as I/O, printing, firmware uploading and other user facing jobs on the new family of \textit{MakerBot} Desktop 3D Printers.  Interfaces with the desktop \textit{MakerWare} software stack for remote printing.
\item[-] Primary developer on \textit{conveyor}, the current backend server for the \textit{MakerWare} software stack.  \textit{conveyor} takes print job parameters from the \textit{MakerWare} GUI, and dispatches to sub-components ranging from the \textit{MiracleGrue} slicer to the \textit{s3g} print driver.
\item[-] Primary developer on \textit{s3g} printer driver and machine code utility package for use with \textit{MakerWare} (https://github.com/makerbot/s3g).
\item[-] Exercised meticulous test driven development protocol; printer driver has roughly 100\% code coverage.
\item[-] Previously Software Technician (Jan 2012 -- Apr 2012), where I wrote extensive testing and QA documentation for MakerBot software and hardware.
\end{itemize}

%----------------------------------------------------------------------------------------

\end{resume}
\end{document}
